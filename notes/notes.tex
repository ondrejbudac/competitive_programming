\documentclass[a4paper, twoside,openany]{book}



%\addtolength{\oddsidemargin}{-.875in}
\addtolength{\evensidemargin}{-1.2in}
\addtolength{\textwidth}{1.2in} %{1.75in}
\addtolength{\topmargin}{-.875in}
\addtolength{\textheight}{1.75in}

\setlength{\parskip}{0.3em}
\setlength{\parindent}{0pt}
%----------------------------------------------------------------------------------------
\usepackage[T1]{fontenc}
\usepackage{lmodern}

\usepackage{listings} % Required for inserting code snippets
\usepackage[usenames,dvipsnames]{color} % Required for specifying custom colors and referring to colors by name
\usepackage{enumitem}
\usepackage{amsmath,amssymb,amsthm}
%\usepackage{showframe}


\usepackage{titlesec}
\titleformat{\chapter}[display]{\normalfont\bfseries\filcenter}{}{1ex}{\titlerule[2pt]\vspace{2ex}\LARGE\thechapter\quad}[\vspace{1ex}{\titlerule[2pt]}]

\definecolor{gray75}{gray}{0.75}
\newcommand{\hsp}{\hspace{20pt}}
%\titleformat{\chapter}[hang]{\Huge\bfseries}{\thechapter\hsp\textcolor{gray75}{|}\hsp}{0pt}{\Huge\bfseries}



\definecolor{DarkGreen}{rgb}{0.0,0.3,0.0} % Comment color
\definecolor{highlight}{RGB}{255,251,204} % Code highlight color

\lstdefinestyle{Style1}{ % Define a style for your code snippet, multiple definitions can be made if, for example, you wish to insert multiple code snippets using different programming languages into one document
language={C++}, % Detects keywords, comments, strings, functions, etc for the language specified
%backgroundcolor=\color{highlight}, % Set the background color for the snippet - useful for highlighting
basewidth  = {.5em,0.5em},
basicstyle=\footnotesize\ttfamily, % The default font size and style of the code
breakatwhitespace=false, % If true, only allows line breaks at white space
breaklines=true, % Automatic line breaking (prevents code from protruding outside the box)
captionpos=b, % Sets the caption position: b for bottom; t for top
columns=fixed,
commentstyle=\footnotesize\ttfamily\color{DarkGreen}, % \usefont{T1}{pcr}{m}{sl}\color{DarkGreen}, % Style of comments within the code - dark green courier font
deletekeywords={}, % If you want to delete any keywords from the current language separate them by commas
%escapeinside={\%}, % This allows you to escape to LaTeX using the character in the bracket
firstnumber=1, % Line numbers begin at line 1
frame=leftline, % Frame around the code box, value can be: none, leftline, topline, bottomline, lines, single, shadowbox
frameround=ffff, % Rounds the corners of the frame for the top left, top right, bottom left and bottom right positions
keywordstyle=\footnotesize\ttfamily\color{Green}\bfseries, % Functions are bold and blue
morekeywords={__int64}, % Add any functions no included by default here separated by commas
numbers=left, % Location of line numbers, can take the values of: none, left, right
numbersep=10pt, % Distance of line numbers from the code box
numberstyle=\footnotesize\ttfamily\color{Black}, % Style used for line numbers
rulecolor=\color{black}, % Frame border color
showstringspaces=false, % Don't put marks in string spaces
showtabs=false, % Display tabs in the code as lines
stepnumber=1, % The step distance between line numbers, i.e. how often will lines be numbered
stringstyle=\color{Purple}, % Strings are purple
tabsize=2, % Number of spaces per tab in the code
}

% Create a command to cleanly insert a snippet with the style above anywhere in the document
\newcommand{\insertcode}[2]{\begin{itemize}\item[]\lstinputlisting[caption=#2,label=#1,style=Style1]{#1}\end{itemize}} % The first argument is the script location/filename and the second is a caption for the listing

%----------------------------------------------------------------------------------------

\begin{document}

\chapter{Graph algorithms}
%----------------------------------------------------------------------------------------
\section{Floyd-Warshall algorithm}
It is a graph analysis algorithm for finding shortest paths in a weighted graph with positive or negative edge weights with no negative cycles.
A single execution of the algorithm will find the lengths (summed weights) of the shortest paths between all pairs of vertices, though it does not return details of the paths themselves.

The Floyd-Warshall algorithm compares all possible paths through the graph between each pair of vertices.
It is able to do this with $\Theta(|V|^3)$ comparisons in a graph. 
This is remarkable considering that there may be up to $\Omega(|V|^2)$ edges in the graph, and every combination of edges is tested. 
It does so by incrementally improving an estimate on the shortest path between two vertices, until the estimate is optimal. 

\insertcode{"codes/floyd_warshall.cpp"}{A C++ code for the Floyd-Warshall algorithm} % The first argument is the script location/filename and the second is a caption for the listing

To detect negative cycles using the Floyd-Warshall algorithm, one can inspect the diagonal of the path matrix, and the presence of a negative number indicates that the graph contains at least one negative cycle.

\subsection{Minimal path reconstruction} 
With simple modifications, it is possible to create a method to reconstruct the actual path between any two endpoint vertices. 
The Shortest-path tree can be calculated for each node in $\Theta(|E|)$ time using $\Theta(|V|)$ memory to store each tree which allows us to efficiently reconstruct a path from any two connected vertices.

\insertcode{"codes/floyd_warshall_reconstruct.cpp"}{A C++ code for the Floyd-Warshall algorithm with information about the minimal path.} 

\subsection{Widest path problem}
In graph algorithms, the \emph{widest path problem}, also known as the \emph{bottleneck shortest path problem} or the \emph{maximum capacity path problem}, is the problem of finding a path between two designated vertices in a weighted graph, maximizing the weight of the minimum-weight edge in the path.
The Floyd-Warshall algorithm can be applied to solve this problem with minor modifications.
One has to reverse the inequality and the addition operation is replaced by the minimum operation.

\insertcode{"codes/floyd_warshall_routing.cpp"}{A C++ code for the Floyd-Warshall algorithm used for optimal routing in a constrained network.} 

%----------------------------------------------------------------------------------------
\chapter{Practical tricks}

\section{Bitmask tricks}

Integers can represent sets. Let $N \geq 0$ and cosider all integers in $\{ 0, 1, 2, \ldots, 2^N-1 \}$. If every place of their binary representation (of length at most $N$) is considered an indicator of an element, they can decode all possible subsets of a set with $N$ elements. For example, if $N=8$ and we consider a set $\{ 0, 1, \ldots, 7\}$, then the number $A=00100101_2$ (binary) encodes the set $\{ 0, 2, 5 \}$. We can do the following set manipulation using bit operations:

\begin{tabular}{ll}
{\lstinline[style=Style1]!A & B!} & represents set intersection\\
{\lstinline[style=Style1]!A | B!} & represents set union\\
{\lstinline[style=Style1]!A ^ B!} & represents set xor (exclusive or)\\
{\lstinline[style=Style1]!((1 << N) - 1) ^ A!} & represents set complement\\
{\lstinline[style=Style1]!A & ~(A-1)!} & is the singleton set with the smallest element of {\lstinline[style=Style1]!A!}\\
{\lstinline[style=Style1]!~(A & (A-1))!} & is a test if the set is singleton (power of $2$)\\
{\lstinline[style=Style1]!A |= 1 << i!} & adds an element\\
{\lstinline[style=Style1]!A &= ~(1 << i)!} & removes an element\\
{\lstinline[style=Style1]+(A & 1 << i) != 0+} & tests if an element is in the set\\
\end{tabular}

To get the greatest element of {\lstinline[style=Style1]!A!} one can use a function.

\insertcode{"codes/highest_set_bit.cpp"}{Find the highest set bit of a given bitmask.} 

To iterate over all subsets of the set ${0, 1, 2, \ldots, N-1}$ we can use the following code.

\insertcode{"codes/list_subsets.cpp"}{A bitmask trick to list all the subsets without recursion.} 

To iterate over all subsets of a set representet by a bitmask we can modify this algorithm to get 

\insertcode{"codes/list_subsets_of_bitmask.cpp"}{A bitmask trick to list all the subsets of a set given by a bitmask without recursion.}

When dealing with 64-bit bitmasks it is necessary to be careful with the types and use {\lstinline[style=Style1]!1LL!} instead of {\lstinline[style=Style1]!1!} in all the circumstances. Indeed {\lstinline[style=Style1]!1 << 62 = 0!} and {\lstinline[style=Style1]!1LL << 62 = 4611686018427387904!}.

%----------------------------------------------------------------------------------------

\section{Using STL}
This section will feature many tips and tricks and notes about the STL containers and algorithms.

\insertcode{"codes/STL.cpp"}{Introduction to STL}

%----------------------------------------------------------------------------------------
\chapter{Linear algebra}

\section{Gaussian elimination}
Given a set of non-negative integers $\{ a_0, a_1, \ldots, a_{N-1} \}$, find a subset that has the highest value when bitwise xor is applied to all its elements.
We perform a complete Gaussian elimination and then xor the basis vectors to get the maximum.
We keep the track of the inverse transformation of the basis in bitmasks to easily reconstruct the original subset

\insertcode{"codes/maximal_xor.cpp"}{An algorithm to find a subset of a given set with maximal bitwise xor using Gaussian elimintaion.}

\chapter{Classical problems}
\section{Longest common subsequence problem}
The longest common subsequence (LCS) problem is to find the longest subsequence common to all sequences in a set of sequences (often just two).
When only two sequences $\{ a_0, a_1, \ldots, a_{n-1}\}$ and $\{ b_0, b_1, \ldots, b_{m-1}\}$ are considered, there is a dynamic approach that solves this problem in $O(nm)$.
If $L(i,j)$ is the length of the longest common subsequence to $\{ a_0, a_1, \ldots, a_{i-1}\}$ and $\{ b_0, b_1, \ldots, b_{j-1}\}$, then we have
\begin{equation}
L(i,j) = \begin{cases} L(i-1,j-1) &  \textnormal{if } a_i = b_j\\ \max(L(i,j-1), L(i-1,j)) & \textnormal{otherwise.} \end{cases}
\end{equation}
This leads to a simple algorithm that uses only $O(\min(m,n))$ memory.




\end{document}